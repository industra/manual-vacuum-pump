\documentclass[fleqn]{article}


\usepackage[utf8]{inputenc}
\usepackage[T1]{fontenc}
\usepackage{industra}
\usepackage{tabularx}
\usepackage{txfonts}


\usepackage{brochure-venturis}
\usepackage{qrcode}
\def\wLaTeX{\qrcode{https://labs.industra.space/wiki/V\%C3\%BDv\%C4\%9Bva}}

\begin{document}
	


\noindent\begin{tabular}{
  @{}%	                   flush left margin
  b{.35\columnwidth}%	   logo
  @{\hspace{.03\columnwidth}}%        gap
  >{\huge\centering\color{DarkBlue}}p{.62\columnwidth}%	   headline
  @{}%                     flush right margin
}
  \raisebox{-35pt}{%
    \fontsize{60}{0}\selectfont\color{Black}\bfseries\wLaTeX}
&
  Industra Labs\\[4pt]\hrule\vspace*{7pt} 
  
  \par
  \fontsize{16}{18}\selectfont\itshape
  Uživatelský manuál -- vývěva
\end{tabular}

\bigskip\bigskip

\noindent\begin{tabular}{@{}
                         p{.34\columnwidth}%		blurb
		         @{\hspace{.04\columnwidth}}% 	gap
		         p{.62\columnwidth}%		quotes
		         @{}%			implicit margin
}
\rmfamily\lite\fontsize{14}{15}\selectfont\fontencoding{T1}\raggedright 


\textit{Více informací na wiki https://labs.industra.space}
\bigskip\\
\textbf{Na co si dát pozor:}
\begin{itemize}[noitemsep,topsep=0pt]
	\item zkontrolovat neporušenost víka,
	\item před spuštěním pumpy odstranit krytku výpusti,
	\item při zapnutí/vypnutí pumpy jsou obě páčky v~poloze OFF,
	\item při vpouštění vzduchu zpět do komory musí být páčka u hadice v~poloze OFF,
	\item nenechávat v podtlaku příliš dlouho, hrozí deformace/prasknutí víka.
\end{itemize}
\bigskip
\textbf{Zakázané materiály:}\\
\begin{itemize}[noitemsep,topsep=0pt]
	\item materiály extrémně hořlavé nebo výbušné při nižším tlaku,
	\item prašné materiály.

\end{itemize}
\bigskip

\textbf{Podpora}\\
\begin{itemize}[noitemsep,topsep=0pt]
	\begin{itshape}
		\item simon@vaizard.net
		\item ondrej@industra.space
	\end{itshape}
\end{itemize}


\par\bigskip
%\vbox to7pc{\hrule\hbox to\hsize{\vrule height7pc\hfill\vrule}\hrule}

&\large
% front RH blurb
\lettrine[lines=1]{}{Vývěva} slouží pro vytvoření podtlaku v komoře (nádobě). Může sloužit k různým účelům, například pro odstranění vzduchových bublin z epoxidové pryskyřice.

\subsection{Poznámky a upozornění}
\begin{itemize}[noitemsep,topsep=0pt]
	\item Nepoužívejte, pokud je víko deformované nebo prasklé.
	\item Pumpa je relativně hlasitá a její zvuk je jiný při chodu \uv{na prázdno} a při odsávání.
	\item Při manipulaci vždy udržujte víko ve vodorovné poloze a neklopte jej. Olej z barometru může vytékat.
	\item Vždy použijte \uv{vlastní dno} -- kus dřeva, plexisklo nebo silikonovou podložku.
\end{itemize}

\subsection{Použití}
\begin{itemize}[noitemsep,topsep=0pt]
	\item Pumpu a komoru přeneste na stabilní podklad a připojte do elektrické sítě. Víko je s pumpou pevně spojeno, nepokoušejte se jednotlivé části odpojit.
	\item Položte váš výrobek do komory a na komoru položte víko.
	\item Zkontrolujte hladinu oleje v barometru (je téměř plný oleje).
	\item Obě páčky musí být v poloze OFF.
	\item Odstraňte krytku výpusti na pumpě (\textit{exhaustion cap}, označen žlutým textem).
	\item Pumpu zapněte a postupným otevíráním ventilu (páčky) blíže u hadice začněte odsávat vzduch z komory.
	\item Pokud během 10 sekund nedojde k poklesu tlaku, dejte páčku plně do polohy ON a víko ke komoře lehce přitiskněte, dokud tlak nezačne klesat, poté se víko udrží samo.
	\item Po dosažení požadovaného (pod)tlaku přepněte páčku zpět do polohy OFF. Pokud budete po chvíli znovu vzduch odsávat, pumpu můžete nechat zapnutou. Jinak pumpu vypněte.
	\item Pro obnovu tlaku pomalu přesuňte druhou páčku (tu vzdálenější od hadice) do polohy ON. Druhá musí zůstat v poloze OFF, jinak hrozí poškození pumpy.
	\item Po dokončení práce vraťte pumpu a komoru zpět na své místo.
\end{itemize}



\end{tabular}

\vfill

\begin{figure}[b]
	\flushright
	\begin{itshape}
    revision: ---REVISION\_ID---, release: ---RELEASE\_ID---, build date: ---BUILDDATE---
	\end{itshape}
\end{figure}



\end{document}
